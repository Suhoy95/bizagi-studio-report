\chapter{Установка Bizagi Studio}

Первым этапом в ознакомлении с BPM-системами, встающим при начале разработки,
является установка соответствующих инструментов. В нашем случае, мы
можем скачать устоновочную программу с официального сайта
\textit{Bizagi}(\url{https://www.bizagi.com/}) в ознакомительных целях.

Сама установка программных компонент является тривиальной последовательнустью
действий, выполняемых с помощью мастера установки. Единственной проблемой,
которая может возникнуть -- это специальные требования к СУБД, которые
хорошо описаны в документации к Bizagi Studio
\cite[SQL Server requisites]{sql-requisites}: потребуется установить базу
данных \textit{MS SQL EXPRESS} (версии описанной в документации), а так же
сустему управления этой базой данных (\textit{MS SQL Managenment Studio}),
чтобы провести необходимую предварительныю настройку (разрешение подключения к БД
с помощью логина\\пароля и создание учетной записи администратора для Bizagi Studio).

В дальнейшем нам достаточно лишь запустить Bizagi Studio (Рис. \ref{20-bizagi-studio}),
задать имя у нового проекта (Рис. \ref{20-create-project}) и описать подключение
к настроенной базе данных (Рис. \ref{20-database}). И если все указано верно,
мы увидим карусель разработки web-приложения (Рис. \ref{20-carrousel}).

\myImage{Приветствующий экран Bizagi Studio}{20-bizagi-studio}{20-bizagi-studio}
\myImage{Создание нового проекта}{20-create-project}{20-create-project}
\myImage{Указание настроек подключения к БД}{20-database}{20-database}
\myImage{Карусель разработки созданного проекта}{20-carrousel}{20-carrousel}
