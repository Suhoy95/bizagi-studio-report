\chapter{Разработка инфологической модели для БП}

Описав бизнес процесс можно перейти к следующему этаапу. Повернув карусель разработки
вправо, мы перейдем к пункту <<Model data>>. В нем можно перйти к описанию
диограмм сущностей для баз данных. Чтобы добавить основную информацию к нашему процессу,
выберем пункт <<Properties>> в выпадающем меню (Рис. \ref{40-properties}).

\myImage{Выбираем пункт <<properties>>}{40-properties}{40-properties}
\clearpage

Там, мы можем подкорректировать видимое название у нашего рпоцесса, а также в
аттрибутах мы опишем основные данные, которые будут заполняться при работе WEB-приложения:
Имя и Фамилия заявителя, контактный телефон для связи, а также поля для
сканов всех необходимых документов участвующих в нашей услуге.

\myImage{Заполняем аттрибуты нашей услуги -- все необходимые документы}{40-attributes}{40-attributes}
