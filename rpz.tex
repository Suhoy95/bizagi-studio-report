%% Преамбула TeX-файла

% 1. Стиль и язык
\documentclass[utf8x]{G7-32} % Стиль (по умолчанию будет 14pt)
\usepackage[T2A]{fontenc}
\usepackage[russian]{babel}
% Остальные стандартные настройки убраны в preamble.inc.tex.
\include{preamble.inc}

% Настройки листингов.
\include{listings.inc}

% Полезные макросы листингов.
\include{macros.inc}
\usepackage[figuresleft]{rotating}

\begin{document}

\pagestyle{empty}
\begin{center}
    Министерство образования и науки Российской Федерации\\
    ФГАОУ ВПО  «УрФУ имени первого Президента России Б. Н. Ельцина»\\
    Институт радиоэлектроники и информационных технологий - РтФ\\
    Департамент информационных технологий и автоматики
    \par
    \vspace{3.5cm}
    \Large{
      Разработка WEB-систем бизнес процесс менеджмента
      на базе BPM платформы Bizagi

      \par
      \vspace{0.5cm}

      ОТЧЕТ\\
      по лабораторной работе
    }

    \vspace{3.5cm}
    {
      Преподаватель: \hfill Клебанов Борис Исаевич
    }
    \par
    {
      Студент: \hfill Сухоплюев Илья Владимирович
    }
    \par
    {
              \hfill Неволин Роман Дмитриевич
    }
    \par
    {
      Группа: \hfill РИ-440001
    }

    \par
    \vspace{3cm}
    Екатеринбург\\
    2017
\end{center}


\frontmatter % выключает нумерацию ВСЕГО; здесь начинаются ненумерованные главы: реферат, введение, глоссарий, сокращения и прочее.

% Команды \breakingbeforechapters и \nonbreakingbeforechapters
% управляют разрывом страницы перед главами.
% По-умолчанию страница разрывается.

% \nobreakingbeforechapters
% \breakingbeforechapters

\pagestyle{plain}

\tableofcontents


\Introduction

Целью данной лабораторной работы является ознакомление с современными средствами
управления бизнес-процессами (Business Process Management, BPM - системы).
Знакомство с данными инструментами позволяет быстро проектировать и разрабатывать
бизнес-приложение благодаря наглядному высокоуровневому языку (BPMN-нотацией), что
в случае разработки классическими средствами требует значительного
количество временных, лоюдских и денежных ресурсов.

В качестве BPM-системы будет изучена платформа Bizagi Studio, в рамках которой
мы создадим простое WEB-приложение для автоматизации оказания государственной
услуги \textit{<<Подготовка предложений о представле­нии к награждению знаком
отличия Свердловской области <<овет да любовь>>}(Далее, услуга Совет да любовь).
Что позволит ознакомится с основными инструментальными средствами BPM-систем\cite{method}:

\begin{itemize}
    \item Cредства проектирования: Графический дизайнер бизнес процесса (БП),
    конструктор базы данных (БД);
    \item Cредства исполнения БП: <<движок>> (BPM Engine),
    исполняющий описанный БП;
    \item Cредства мониторинга БП.
\end{itemize}


\mainmatter % это включает нумерацию глав и секций в документе ниже

\chapter{Установка Bizagi Studio}

Первым этапом в ознакомлении с BPM-системами, встающим при начале разработки,
является установка соответствующих инструментов. В нашем случае, мы
можем скачать устоновочную программу с официального сайта
\textit{Bizagi}(\url{https://www.bizagi.com/}) в ознакомительных целях.

Сама установка программных компонент является тривиальной последовательнустью
действий, выполняемых с помощью мастера установки. Единственной проблемой,
которая может возникнуть -- это специальные требования к СУБД, которые
хорошо описаны в документации к Bizagi Studio
\cite[SQL Server requisites]{sql-requisites}: потребуется установить базу
данных \textit{MS SQL EXPRESS} (версии описанной в документации), а так же
сустему управления этой базой данных (\textit{MS SQL Managenment Studio}),
чтобы провести необходимую предварительныю настройку (разрешение подключения к БД
с помощью логина\\пароля и создание учетной записи администратора для Bizagi Studio).

В дальнейшем нам достаточно лишь запустить Bizagi Studio (Рис. \ref{20-bizagi-studio}),
задать имя у нового проекта (Рис. \ref{20-create-project}) и описать подключение
к настроенной базе данных (Рис. \ref{20-database}). И если все указано верно,
мы увидим карусель разработки web-приложения (Рис. \ref{20-carrousel}).

\myImage{Приветствующий экран Bizagi Studio}{20-bizagi-studio}{20-bizagi-studio}
\myImage{Создание нового проекта}{20-create-project}{20-create-project}
\myImage{Указание настроек подключения к БД}{20-database}{20-database}
\myImage{Карусель разработки созданного проекта}{20-carrousel}{20-carrousel}

\chapter{Импорт BPMN-схемы услуги}

Теперь перейдем к первому этапу проектирования нашего приложения
для услуги Совет да любовь - описания этого бизнес процесса в виде
графического представления в специальной нотации (BPMN).

В нашем случае, ознокомление с данной натацией было проведено в предудущей
лабораторной работе, и нам достаточно просто импортировать готовую схему
нашего процесса в проект (Рис. \ref{30-import}).

\myImage{Выбираем пункт <<Import process>>,
чтобы добавить готовую BPMN-диаграмму}{30-import}{30-import}

После чего мы можем в открывшемся окне \textit{Bizagi Modeler} подкорректоровать
при необходимости нашу схему (Рис \ref{30-description}). На схеме как раз
отображаются основные моменты проведения услуги: гражданин РФ, имеющий быть
честь предоставленный к награде <<Совет да любовь>>, может подать заявление
в многофункциональный центр (МФЦ), после чего в ходе некоторых проверок
(уточнение сведений о судимости в Информационном центре (ИЦ) и проверке соблюдения
прав детей заявителя) Министерство социальной политики заполняет наградной лист
и формулирует предложения о нагрождении, после чего передает все это в Правительство.

\begin{sidewaysfigure}
    \centering
    \includegraphics[width=\textwidth]{figures/30-BPMN}
    \caption{Улуга <<Совет да любовь>> в нотации BPMN}
    \label{30-description}
\end{sidewaysfigure}
\chapter{Разработка инфологической модели для БП}

Описав бизнес процесс можно перейти к следующему этаапу. Повернув карусель разработки
вправо, мы перейдем к пункту <<Model data>>. В нем можно перйти к описанию
диограмм сущностей для баз данных. Чтобы добавить основную информацию к нашему процессу,
выберем пункт <<Properties>> в выпадающем меню (Рис. \ref{40-properties}).

\myImage{Выбираем пункт <<properties>>}{40-properties}{40-properties}
\clearpage

Там, мы можем подкорректировать видимое название у нашего рпоцесса, а также в
аттрибутах мы опишем основные данные, которые будут заполняться при работе WEB-приложения:
Имя и Фамилия заявителя, контактный телефон для связи, а также поля для
сканов всех необходимых документов участвующих в нашей услуге.

\myImage{Заполняем аттрибуты нашей услуги -- все необходимые документы}{40-attributes}{40-attributes}

\chapter{Разработка пользовательского интерфейса}

Описав основные документы, которые необходимо заполнить при предоставлении услуги,
нам требуется описать, как они будут заполнятся.

К счастью, в Bizagi Studio есть полноценный редактор форм, с помощь которого
можно создать необходимые формы в несколько кликов мыши.

\myImage{Перейдя в <<Define Forms>>, нам нужно выбрать для какого
действия нужно создать формы (помечены восклицательным знаком)}{50-add-form}{50-add-form}
\myImage{Перетаскивая аттрибуты данных на форму,
можем легко сформировать панель для регистрации инструментов
(Красным подсвечены поля обязательные для заполнения на данном этапе)
}{50-register-form}{50-register-form}
\myImage{Отправка запроса в ИЦ - не может быть автоматизирвоана по внешним причинам,
поэтому просто дадим исполнителю необходимую информацию и добавим подтверждение
об отправке заявления в ИЦ}{50-request}{50-request}

\myImage{Внесение сведений о судимости в модель.
На этом этапе остальные поля уже не меняются}{50-journal}{50-journal}
\begin{figure}[!ht]
    \centering
    \includegraphics[width=0.5\textwidth]{figures/50-edit}
    \caption{При необходимости, можно отредактировать модель прямо в редакторе форм}
    \label{50-edit}
\end{figure}
\chapter{Настройка бизнес-правил шлюзов}

Для целостной работы форм вместе, нем осталось описать логику работы
шлюза в зависимости от наличия свидетельства о судимости на предыдущем этапе
работы.


\myImage{Выбираем нужный шлюз, а после ветку для которой бедм задавать условие}{60-choose}{60-choose}
\myImage{Переход по ветку <<да>> будет при выполнении условия}{60-yes}{60-yes}
\myImage{Описание условия перехода: Сведения о судимости заполнены на предыдущем шаге}{60-condition}{60-condition}
\myImage{Переход на ветку <<нет>> пройдет в обратном случае}{60-no}{60-no}
\include{65-events}
\chapter{Проверка работы приложения}

\myImage{Нажимаем кнопуку <<Run>>, после чего приложение открывается в браузере}{70-run}{70-run}
\myImage{Интерфейс Bizagi-приложения, запускаем созданный процесс}{70-engine}{70-engine}
\myImage{Заполним форму, убедимся что перейдем к отправке запроса в ИЦ}{70-registration}{70-registration}
\myImage{При отправке запроса видим наше требование на подтверждение}{70-confirm}{70-confirm}
\clearpage
Все остальные шаги процесса выполняются аналогично. Можно попробовать все случаи
и подкорректировать формы и\\или данные для комфортной работы.


\backmatter %% Здесь заканчивается нумерованная часть документа и начинаются ссылки и
            %% заключение

\Conclusion % заключение к отчёту

В результате выполненой работы, было созданно WEB-приложение,
позволяющее автоматизировать государтственную услугу <<Совет да Любовь>>.

В ходе создания данного рпиложения, было проведено знакомство с BPM-системой
Bizagi Studio. Рассмотрены его основные компоненты: будь то графичиские
интерфейсы для описания бизнес процесса, редакторы информационных сущностей
модели и редокторы WEB-форм.

После этого описанне WEB-приложение было запущено в Bizagi Engine и проверена
его работоспособность и удовлетворение основных потребностей в оказании услуги.



%%% Local Variables:
%%% mode: latex
%%% TeX-master: "rpz"
%%% End:


\include{81-biblio}

\appendix   % Тут идут приложения


\end{document}

%%% Local Variables:
%%% mode: latex
%%% TeX-master: t
%%% End:
